\documentclass{article}


\usepackage[english]{babel}
\usepackage{listings}
\usepackage[letterpaper,top=2.5cm,bottom=2.5cm,left=2.5cm,right=2.5cm,marginparwidth=1.75cm]{geometry}

\usepackage{amsmath,amsthm,amssymb}
\usepackage{bbold}
\usepackage{graphicx}
\usepackage{listings}
\usepackage{multicol}
\usepackage{setspace}
\usepackage{enumitem}


%common set symbols 
\newcommand{\Z}{\mathbb{Z}}
\newcommand{\N}{\mathbb{N}}
\newcommand{\R}{\mathbb{R}}
\newcommand{\Q}{\mathbb{Q}}
\newcommand{\C}{\mathbb{C}}
\newcommand{\nullset}{\varnothing}
\newcommand{\powerset}[1]{\mathcal{P} ({#1})}
\newcommand{\ol}[1]{\overline{#1}}

\newcommand{\divides}{\mid}
\newcommand{\notdivides}{\nmid}

\newcommand{\st}{\text{ such that }}
\newcommand{\pmi}{\text{ principle of mathematical induction }}

\newcommand{\nitem}[1] % sets enumerate to a specific number {arg1}
{
	\setcounter{enumi}{#1}
	\addtocounter{enumi}{-1}
	\item
}

\title{MATH 405: Assignment 2}
\author{Micah Sherry}
\begin{document}
	\maketitle
	\begin{enumerate}
		\item Let $m$ and $n$ be positive integers. Prove that if $m\divides n$ and $n\divides m$ then $m=n$.
		\begin{proof}
			Assume $m\divides n$ and $n\divides m$. 
			Since $m\divides n$, $n = m(x_1)$ for some $x_1 \in \Z$ and since $n\divides m$, $m = n(x_2)$ for some $x_2 \in \Z$. 
			Therefore,
			\begin{align*}
				n &= m(x_1)\\
				  &= n(x_2x_1) & \text{(by substitution)}		  
			\end{align*}
			So, $(x_2x_1) = 1$, and because n and m are both positive $x_1=x_2=1$. This implies n=m, which is what we wanted to show. 
		\end{proof}
			\item let $m, n \in \Z$ if $n\divides m$ then $m\Z \subseteq n\Z$.
		\begin{enumerate}
			\item Example: $n=2$ and $m=4$\\
				Note $2 \divides 4$.
				Let $x \in 4\Z$ then $x = 4a$ for some $a \in \Z$. $x=2(2a)$ therefore $x \in 2\Z$ 
			\item prove the statement. 
			\begin{proof}
				Let $n,m \in \Z$ and assume $n \divides m$. 
				Since $n \divides m$, $m = n(a)$ for some $a \in \Z$. Let $x \in m\Z$ so, $x =m(b)$ for some $b \in \Z$.
				So, 
				\begin{align*}
				 x &= m(b) \\
				   &= n(a)(b) &\text{(by substitution)}\\
				   &= n(ab)
				\end{align*}
				Since $ab$ is an integer (because $\Z$ is closed under multiplication) $x$ is an element of $n\Z$. Therefore the subset relationship holds, which is what we wanted to show.   
			\end{proof}				
		\end{enumerate}
		
		\item Prove the converse of the statement above. That is Prove: 
		Let $m,n \in \Z$. If $m\Z \subseteq n\Z$, then $n \divides m$.    
		\begin{proof}
			The proof will be by contradiction.\\
			Assume that $m\Z \subseteq n\Z$ and $n \notdivides m$.
			Notice $m \in m\Z$ (because $m = m(1)$). This means that m has to be an element of $n\Z$ (otherwise it would not be a subset).
			Therefore $m = n(a)$ for some $a \in \Z$ contradicting the assumption. Therefore the original statement is true.
		\end{proof}
		\item For each pair of integers $a$ and $b$ determine whether or not $ax+by=1$ has an integer solution. If it does then find at least 3 different integer solutions $(x,y)$ if it doesn't then explain why not
		\begin{enumerate}
			\item $a=7, b= 10$\\
				Integer solutions $(x,y): (-7, 5 ),\ (3, -2 ),\ (13, -9)$

			\item $a=8, b= 10$\\
				\begin{align*}
					ax+by &= 8x+10y\\
						  &= 2(4x+5y) 
				\end{align*}
				Which implies that all (integer) linear combos of a and b must be even. Therefore there is no integer solutions. 
			\item $a=15, b= 21$ 
			\begin{align*}
				ax+by &= 15x + 21y\\
					  &= 3(5x+7y)
			\end{align*}
			Which implies that all (integer) linear combos of a and b must be a multiple of 3. Therefore there is no integer solutions.
			\item $a=15, b= 16$ \\
			Integer solutions $(x,y): (-1,1), \  (15, -14), \ (-17, 16)$
		\end{enumerate}
		\item Let $a$ and $b$ be consecutive odd integers. show that the $gcd(a,b)=1$.
		\begin{proof}
			Since $a$ and $b$ are odd integers $a = 2k +1$ and $b = 2k+3$ for some $k \in \Z$. Let $d = gcd(a,b)$. 
			Therefore $a = dn$ for some $n \in \Z$ and similarly $b = dm$ for some $m \in \Z$. So,
			\begin{align*}
					b - a &= dm-dn\\
					b - a &= d(m-n) \\
			(2k+3)-(2k+1) &= d(m-n)\\
			            2 &= d(m-n)
			\end{align*}
			Therefor $d \divides 2$ and since a and b are odd $d \not=2$. So, $d = 1$.
		\end{proof}
	\end{enumerate}
	
	\textbf{Extra Credit:}
		prove that $3, 5, 7$ is the only prime triple.
		\begin{proof}
			Let $n\in \Z$ and $n > 3$. Notice that if $n \equiv_3 0 $ it is a multiple of 3, therefore it is not prime.
			\begin{enumerate}[label= Case \arabic*:]
				\item n is even. This is trivial if n is even then it is not prime and therefore not in a prime triple
				\item n is odd. \\  
				Consider the three consecutive odd integers $n, n+2 n+4$. Notice there are three sub-cases for n: $n \equiv_3 0$, $n \equiv_3 1$, $n \equiv_3 2$
				\begin{enumerate}[label= Subcase \arabic*:]
				\item $n \equiv_3 0$. Therefore n is not prime so it is not part of a prime triple.
				\item $n \equiv_3 1$. which implies $n+2 \equiv_3 0$ Therefore $n+2$ is not prime so it is not part of a prime triple.
				\item $n \equiv_3 2$. which implies $n+4 \equiv_3 0$ Therefore $n+4$ is not prime so it is not part of a prime triple.
				\end{enumerate}	
			\end{enumerate}
			Therefore since in all cases for $n$, $n+2$ and $n+4$ is not a prime triple 3, 5 ,7 is the only prime triple.
			
		\end{proof}
			
\end{document}