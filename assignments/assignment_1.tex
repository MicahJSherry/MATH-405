\documentclass{article}


\usepackage[english]{babel}
\usepackage{listings}
\usepackage[letterpaper,top=2.5cm,bottom=2.5cm,left=2.5cm,right=2.5cm,marginparwidth=1.75cm]{geometry}

\usepackage{amsmath,amsthm,amssymb}
\usepackage{bbold}
\usepackage{graphicx}
\usepackage{listings}
\usepackage{multicol}
\usepackage{setspace}
\usepackage{enumitem}


%common set symbols 
\newcommand{\Z}{\mathbb{Z}}
\newcommand{\N}{\mathbb{N}}
\newcommand{\R}{\mathbb{R}}
\newcommand{\Q}{\mathbb{Q}}
\newcommand{\C}{\mathbb{C}}
\newcommand{\nullset}{\varnothing}
\newcommand{\powerset}[1]{\mathcal{P}({#1})}
\newcommand{\ol}[1]{\overline{#1}}

\newcommand{\divides}{\mid}


\newcommand{\st}{\text{ such that }}
\newcommand{\pmi}{\text{ principle of mathematical induction }}

\newcommand{\nitem}[1] % sets enumerate to a specific number {arg1}
{
	\setcounter{enumi}{#1}
	\addtocounter{enumi}{-1}
	\item
}

\title{MATH 405: Assignment 2}
\author{Micah Sherry}
\begin{document}
	\maketitle
	\begin{enumerate}
		\item Let $m$ and $n$ be positive integers. Prove that if $m\divides n$ and $n\divides m$ then $m=n$.
		\begin{proof}
			Assume $m\divides n$ and $n\divides m$. 
			Since $m\divides n$, $n = m(x_1)$ for some $x_1 \in \Z$ and since $n\divides m$, $m = n(x_2)$ for some $x_2 \in \Z$. 
			Therefore,
			\begin{align*}
				n &= m(x_1)\\
				  &= n(x_2x_1) & \text{(by substitution)}		  
			\end{align*}
			So, $(x_2x_1) = 1$, and because n and m are both positive $x_1=x_2=1$. This implies n=m, which is what we wanted to show. 
		\end{proof}
		\item let $m, n \in \Z$ if $n\divides m$ then $m\Z \subseteq n\Z.$  
		\begin{enumerate}
			\item Example: $n=2$ and $m=4$\\
				Note $2 \divides 4$.
				Let $x \in 4\Z$ then $x = 4a$ for some $a \in \Z$. $x=2(2a)$ therefore $x \in 2\Z$ 
			\item prove the statement. 
			\begin{proof}
				Let $n,m \in \Z$ and assume $n \divides m$. 
				Since $n \divides m$, $m = n(a)$ for some $a \in \Z$. Let $x \in m\Z$ so, $x =m(b)$ for some $b \in \Z$.
				So, 
				\begin{align*}
				 x &= m(b) \\
				   &= n(a)(b) &\text{(by substitution)}\\
				   &= n(ab)
				\end{align*}
				Since $ab$ is an integer (because $\Z$ is closed under multiplication) $x$ is an element of $n\Z$. Therefore the subset relationship holds, which is what we wanted to show.   
			\end{proof}				
		\end{enumerate}
		\item 
	\end{enumerate}
\end{document}