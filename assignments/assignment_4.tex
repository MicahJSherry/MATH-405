\documentclass{article}


\usepackage[english]{babel}
\usepackage{listings}
\usepackage[letterpaper,top=2.5cm,bottom=2.5cm,left=2.5cm,right=2.5cm,marginparwidth=1.75cm]{geometry}

\usepackage{amsmath,amsthm,amssymb}
\usepackage{bbold}
\usepackage{graphicx}
\usepackage{listings}
\usepackage{multicol}
\usepackage{setspace}
\usepackage{enumitem}


%common set symbols 
\newcommand{\Z}{\mathbb{Z}}
\newcommand{\N}{\mathbb{N}}
\newcommand{\R}{\mathbb{R}}
\newcommand{\Q}{\mathbb{Q}}
\newcommand{\C}{\mathbb{C}}
\newcommand{\nullset}{\varnothing}
\newcommand{\powerset}[1]{\mathcal{P} ({#1})}
\newcommand{\ol}[1]{\overline{#1}}

\newcommand{\divides}{\mid}
\newcommand{\notdivides}{\nmid}

\newcommand{\st}{\text{ such that }}
\newcommand{\pmi}{\text{ principle of mathematical induction }}

\newcommand{\nitem}[1] % sets enumerate to a specific number {arg1}
{
	\setcounter{enumi}{#1}
	\addtocounter{enumi}{-1}
	\item
}
\usepackage{multicol}


\title{MATH 405: Assignment 4}
\author{Micah Sherry}
\begin{document}
	\maketitle
	\begin{enumerate}
		\item Find the smallest positive solution to the system of congruences.
		\begin{align*}
			x &\equiv 4 \bmod{7}\\
			x &\equiv 5 \bmod{11}\\
			x &\equiv 2 \bmod{16}\\
			x &\equiv 1 \bmod{19}
		\end{align*}
		
		
		\begin{align*}
			M   &= 7  \cdot 11 \cdot 16 \cdot 19 = 23408\\
			M_1 &= 11 \cdot 16 \cdot 19 = 3344\\
			M_2 &= 7  \cdot 16 \cdot 19 = 2128\\
			M_3 &= 7  \cdot 11 \cdot 19 = 1463\\
			M_4 &= 7  \cdot 11 \cdot 16 = 1232
		\end{align*}
		
		\begin{multicols}{2}	
			\begin{align*}
				M_1b_1 &\equiv1 \bmod{7}\\
				5b_1 &\equiv1 \bmod{7}\\
				1  &= 3(5)-2(7)\\
				\text{Therefore } & b_1 \equiv 3\bmod{7}
			\end{align*}
			\begin{align*}
				M_2b_2 &\equiv1 \bmod{11}\\
				5b_2 &\equiv1 \bmod{11}\\
				1  &= -2(5)-(11)\\
			\text{Therefore } & b_2\equiv-2 \equiv \bmod{11}
			\end{align*}
			
			\begin{align*}
				M_3b_3 &\equiv1 \bmod{16}\\
				7b_3 &\equiv1 \bmod{16}\\
				1  &= 7(7)-3(16)\\
				\text{Therefore } & b_3\equiv 7 \bmod{16}
			\end{align*}
			\begin{align*}
				M_4b_4 &\equiv1 \bmod{19}\\
				16b_4 &\equiv1 \bmod{19}\\
				1  &= 6(16)-5(19)\\
				\text{Therefore } & b_4\equiv 6 \bmod{19}
			\end{align*}
		\end{multicols}
			\begin{align*}
				x_0  &= 4M_1b_1 + 5M_2b_2+ 2M_3b_3 + 1M_4b_4\\
				     &= 4(3344)(3)+5(2128)(9)+ 2(1463)(7)+1(1232)(6)\\
				     &= 163762\\
				     \\
				x_k  &= x_0 + kM  \text{(where k is some integer)}\\
				x_{-6}  &= 23314 \text{ is the smallest positive value for x.}				
			\end{align*}
		
		
		\item Consider the set $\Z_3[i] = \{ a+bi |a,b \in \Z_3 \}$ where $i = \sqrt{-1}$.
		\begin{enumerate}
			\item find all the elements of $\Z_3[i]$. How many are there?\\
			Notice there are 3 choices for a and 3 for b, so $\Z_3[i]$ has 9 elements.\\ 
			$0, 1 , 2, 
			i, 1+i, 2+i, 
			2i, 1+2i, 2+2i $
			
			\item $1+2i \in \Z_3[i] $ has a multiplicative inverse in find it.
			 \begin{align*}
			 	(1+2i)(2+2i) &=\\
			 		&= 2+4i+2i +4i^2 \\
			 		&= 2-4 \\
			 		&= -2 \\
			 		&= 1
			 \end{align*}
			 Therefore $(2+2i)$ is the multiplicative inverse of  (1+2i).
			 
			\item Classify each nonzero element of $\Z_3[i]$ as a unit, a zero divisor or 
			neither. 
			\begin{table}[htbp]
				\centering
				\begin{tabular}{|c|c|c|}
					\hline
					Number & Inverse & classification\\
					\hline
					\hline
					0 & NA & Neither \\
					\hline
					1 & 1 & unit \\
					\hline
					2 & 2 & unit\\
					\hline
					
					i & 2i & unit \\
					\hline
					1+i & 2+i &  unit \\
					\hline
					2+i & 1+i &  unit \\
					\hline
					
					2i & i & unit \\
					\hline
					1 +2i & 2i+2 &  unit \\
					\hline
					2+2i & 1+2i &  unit \\
					\hline
					
				\end{tabular}
			\end{table}
			
		\end{enumerate}
		
		\item let $R$ be a ring and let $S$ and $T$ be subrings or $R$. Let $M = S \cap T$. Show that M is a subring of $R$ 
		\begin{proof}
				Let $S$ and $T$ be subrings of a Ring $R$. And Let $M = S \cap T$.\\ 
				To show M is a subring  we need to show that $ 0, 1 \in M$, $M$ is closed under addition and multiplication, and if $a \in M$ then $-a \in M$ 
				
				Since $S$ is a subring $0 \in S$.
				Similarly, $T$ is a subring $0 \in T$.
				Therefore $0 \in M$.
				
				Since $S$ is a subring $1 \in S$.
				Similarly, $T$ is a subring $1 \in T$.
				Therefore $1 \in M$.
		
				Let $a,b \in M$.
				 Since, $a,b \in S$ since $S$ is a subring $a+b \in S$\\
				 Similarly, $a,b \in T$ since $T$ is a subring $a+b \in T$\\
				Therefore $a+b \in M $ Thus $M$ is closed under addition. 
				
				Let $a,b \in M$.
				Since, $a,b \in S$ since $S$ is a subring $a\cdot b \in S$\\
				Similarly, $a,b \in T$ since $T$ is a subring $a\cdot b \in T$\\
				Therefore $a \cdot b \in M $ Thus $M$ is closed under multiplication.
				
				Let $a \in M$.
				Since, $a\in S$ since $S$ is a subring $-a\in S$\\
				Similarly, $a\in T$ since $T$ is a subring $-a\in T$\\
				Therefore if $a \in M$ then $-a \in M$.
				
				Therefore M is a Subring of R
		\end{proof}
		
		\item Let $R$ be a ring (not necessarily commutative). \\
		Let $a,b \in R$, prove that $(a \cdot b)^{-1} = b^{-1} \cdot a^{-1}$
		\begin{proof}
			Let R be a ring and let $a,b \in R$. And assuming $a^{-1}, b^{-1}$ exist.   
			Consider:
			\begin{align*}
				 (a \cdot b) \cdot (b^{-1} \cdot a^{-1}) &= a \cdot (b \cdot b^{-1}) \cdot a^{-1} & \text{(by associativity of R)}\\
						 						&= a \cdot 1 \cdot a^{-1} & \text{(by definition of multiplicative inverse)}\\
						 						&= a \cdot  a^{-1} & \text{(by definition of multiplicative identity)}\\
						 						&= 1 &\text{(by definition of multiplicative inverse)}
			\end{align*}
			Therefore  $(b^{-1} \cdot a^{-1})$ is the multiplicative inverse of $(a\cdot b)$
		\end{proof}
		
		\textbf{Extra Credit:}
		let R be a ring (not necessarily commutative).\\
		If for any $a,b \in R$, $(a\cdot b)^{-1} = a^{-1} \cdot b^{-1}$ then show that R is commutative.
		\begin{proof}
		Let $a,b \in R$.
		Assume $a^{-1}, b^{-1}$ exist and $(a \cdot b)^{-1} =(a^{-1} \cdot b^{-1})$. from proof of problem 4 we have $(a \cdot b)^{-1} = (b^{-1} \cdot a^{-1})$. 
		So,
		\begin{align*}
			(a^{-1} \cdot b^{-1}) &= (b^{-1} \cdot a^{-1})\\
			 b^{-1} &= a \cdot (b^{-1} \cdot a^{-1})		 &\text{(left multiply by a)}\\
			 	  1 &= b\cdot a \cdot (b^{-1} \cdot a^{-1})  &\text{(left multiply by b)}\\
			 	  a &= b \cdot a \cdot b^{-1} 				 &\text{(right multiply by a)}\\
			 a \cdot b &= b \cdot a                          &\text{(right multiply by b)}\\
		\end{align*}
		Therefore since $b \cdot a = a \cdot b$, R is commutative. 
		\end{proof}
	\end{enumerate}
\end{document}