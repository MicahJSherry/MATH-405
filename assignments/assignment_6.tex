\documentclass{article}


\usepackage[english]{babel}
\usepackage{listings}
\usepackage[letterpaper,top=2cm,bottom=2cm,left=2cm,right=2cm,marginparwidth=1.5cm]{geometry}

\usepackage{amsmath,amsthm,amssymb}
\usepackage{bbold}
\usepackage{graphicx}
\usepackage{listings}
\usepackage{multicol}
\usepackage{setspace}
\usepackage{enumitem}


%common set symbols 
\newcommand{\Z}{\mathbb{Z}}
\newcommand{\N}{\mathbb{N}}
\newcommand{\R}{\mathbb{R}}
\newcommand{\Q}{\mathbb{Q}}
\newcommand{\C}{\mathbb{C}}
\newcommand{\nullset}{\varnothing}
\newcommand{\powerset}[1]{\mathcal{P} ({#1})}
\newcommand{\ol}[1]{\overline{#1}}

\newcommand{\divides}{\mid}
\newcommand{\notdivides}{\nmid}

\newcommand{\st}{\text{ such that }}
\newcommand{\pmi}{\text{ principle of mathematical induction }}

\newcommand{\nitem}[1] % sets enumerate to a specific number {arg1}
{
	\setcounter{enumi}{#1}
	\addtocounter{enumi}{-1}
	\item
}
\usepackage{multicol}


\title{MATH 405: Assignment 6}
\author{Micah Sherry}
\begin{document}
	\maketitle
	\begin{enumerate}
		\item Let $R$ be an integral domain. Prove that if $a, b \in R$ and $a^2 = b^2$, then $a = b$ or $a = -b$.
		\begin{proof}
		
			Let $a,b \in R$ with $a^2 = b^2$.
			So,
			\begin{align*}
				a^2 	   &= b^2\\
				a^2-b^2    &= 0_R\\
				(a-b)(a+b) &= 0_R
			\end{align*}  
			Since $R$ does not have zero divisors $a-b = 0_R$ or $a+b = 0_R$. Therefore $a = b$ or $a = -b$.
		\end{proof}



		\item Let $R$ be a commutative ring (but not necessarily an integral domain). 
		Let $f(x) \in R[x]$ prove each of the following statements. \\ 
		To make the following proofs more concise, we'll prove this statement first:
		if $f(x) \in R[x]$ is monic and non-constant and $0_R \not=g(x) \in R[x]$, Then $\deg(f(x)g(x))>0$.
		\begin{proof}
			Assume,\\$f(x) = 1_Rx^n+ a_{n-1}x^{n-1} + \ldots + a_0$ and\\
			$g(x) = b_mx^m+b_{m-1}x^{m-1} + \ldots + b_0$ with $n>0$, $m \geq 0$, and $a_i, b_i \in R$.\\
			Consider the product of the highest terms of $f(x)g(x)$, $b_m*1_Rx^{m+n}$.
		Since $1_R$ is not a zero divisor $\deg(f(x)g(x)) = n + m >0$.
		\end{proof}

		\begin{enumerate}
			\item If $f(x)$ is monic and non-constant, then $f(x)$ is not a unit in $R[x]$.
			\begin{proof}
				Assume $f(x)$ is monic and non-constant.
				And assume for the sake of contradiction there exist a $g(x)$ such that $f(x)g(x) = 1_R$. 
				From the above result $\deg(f(x)g(x))>0$, contradicting the assumption that $f(x)g(x) = 1_R$ 
				therefore $f(x)$ is not a unit
			\end{proof} 

			\item If $f(x)$ is monic, then $f(x)$ is not a zero divisor in $R[x]$
			\begin{proof}
				assume f(x) is monic.
				\begin{enumerate}[label=Case \arabic*:]
				\item $f(x)$ is monic and constant. $f(x)= 1_R$. Which is not a zero divisor.
				\item f(x) is monic and non-constant.
				Assume for the sake of contradiction that there exists $g(x) \not= 0_R$ such that $f(x)g(x)=0_R$. 
				From the above result the $\deg(f(x)g(x))>0$ which is contradicting the assumption that $f(x)g(x)=0_R$.
				thus $f(x)$ is not a zero divisor.
				\end{enumerate}
				Therefore the original statement holds. 
			\end{proof}
		\end{enumerate}

		\item units in $\Z_4[x]$ 
		\begin{enumerate}
			\item Find five units (other than 1 and 3) in $\Z_4[x]$ 
				\begin{itemize}
					\item $  2x+1$; $(2x  +1)^2 = 4x^2  +4x   +1 = 1$
					\item $2x^2+1$; $(2x^2+1)^2 = 4x^4  +4x^2 +1 = 1$
					\item $2x^3+1$; $(2x^3+1)^2 = 4x^6  +4x^3 +1 = 1$
					\item $2x^4+1$; $(2x^4+1)^2 = 4x^8  +4x^4 +1 = 1$
					\item $2x^5+1$; $(2x^5+1)^2 = 4x^{10} +4x^5 +1 = 1$
				\end{itemize}
			\item Explain why $\Z_4[x]$ has infinitely many units.\\
			Consider all polynomials of the form $(2g(x)+1)$, with $g(x) \in Z_4[x]$.
			Notice that $(2g(x)+1)^2 = 4g(x)^2 +4g(x)+1 = 1$. Therefore, are infinitely many units of the form $(2g(x)+1)$.   
				
		\end{enumerate}

		\item Consider the function $\theta: \Z_2[x] \to \Z_2[x]$ 
		where $\theta(f(x)) = (f(x))^2$ for any $f(x) \in  \Z_2[x]$.
		\begin{enumerate}
		\item show that $\theta$ is a homomorphism.
		\begin{proof}
		Let $f(x),\ g(x) \in \Z_2[x]$
		Consider, 
		\begin{align*}
			\theta(f(x)g(x))&= (f(x) g(x))^2\\
							&= f(x)^2g(x)^2\\
							&= \theta(f(x))\theta(g(x)).
		\end{align*}
		Therefore $\theta$ preserves multiplication.
		Now, consider 
		\begin{align*}
			\theta(f(x)+g(x)) &= (f(x)+g(x))^2\\ 
							&= f(x)^2 + 2f(x)g(x) +g(x)^2 \\
							&= f(x)^2+g(x)^2\\
							&= \theta(f(x))+\theta(g(x))
		\end{align*}
		Therefore $\theta$ preserves addition. 
		Thus $\theta$ is a homomorphism.
		\end{proof}
		\item find $\ker\theta$\\
		$\ker\theta= \{f(x)\in \Z_2[x]| f(x)^2 = 0\}= \{0\}$

		\item Describe all elements in the image of $\theta$\\ 
		Notice $\theta$ maps each term of the polynomial to its Square.
		So the image of $\theta$ is all polynomials in $\Z_2[x]$ with only even powers of x
		
	\end{enumerate}
	\item Let $A$ be the set of all polynomials in $\Z[x]$ with an even constant term.
	\begin{enumerate}
		\item prove that $A$ is an ideal $\Z[x]$
		\begin{proof}
			let $A = \{ xg(x)+2n|g(x)\in \Z[x] \wedge n \in \Z \}$\\
			let $f_1(x) = xg_1(x)+2n_1$ and $f_2(x) = xg_2(x)+2n_2$, 
			for some $g_1(x), g_2(x) \in \Z[x]$, $n_1, n_2 \in \Z$. 
			
			Consider,
			\begin{align*}
				f_1(x)+f_2(x) &= xg_1(x)+2n_1+xg_2(x)+2n_2 \\
							&= x(g_1(x)+g_2(x))+2(n_1+n_2)
			\end{align*}

			So, $A$ is an closed under addition.
			
			let $t(x) = xg(x)+ n$ for some  $g(x) \in \Z[x]$ and $n \in \Z$.
			Now Consider, 
			\begin{align*}
				t(x)f_1(x)  &= (xg(x)+ n)(xg_1(x)+2n_1) \\
							&=x(xg(x)g_1(x)+2n_1g(x)+ng_1(x))+ 2n_1n
			\end{align*}
			This show that A satisfies the absorption property and closure under multiplication.
			Therefore A is an ideal.
		\end{proof}
		\item Anita claims $A = \langle2\rangle =\{ 2*f(x)| f(x) \in Z[x] \}$. Do you agree or
		disagree? Explain.\\
		Disagree, notice $x+2 \in A$ and $x+2 \not\in \langle2\rangle$

		\item Elizabeth claims $A = \langle x \rangle =\{ x*f(x)| f(x) \in Z[x] \}$. Do you agree or
		disagree? Explain.\\
		Disagree, notice $2 \in A$ and $2 \not\in \langle x \rangle$ 
	\end{enumerate}



	\end{enumerate}
\end{document}