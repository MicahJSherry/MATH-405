\documentclass{article}


\usepackage[english]{babel}
\usepackage{listings}
\usepackage[letterpaper,top=2cm,bottom=2cm,left=2cm,right=2cm,marginparwidth=1.5cm]{geometry}

\usepackage{amsmath,amsthm,amssymb}
\usepackage{bbold}
\usepackage{graphicx}
\usepackage{listings}
\usepackage{multicol}
\usepackage{setspace}
\usepackage{enumitem}


%common set symbols 
\newcommand{\Z}{\mathbb{Z}}
\newcommand{\N}{\mathbb{N}}
\newcommand{\R}{\mathbb{R}}
\newcommand{\Q}{\mathbb{Q}}
\newcommand{\C}{\mathbb{C}}
\newcommand{\nullset}{\varnothing}
\newcommand{\powerset}[1]{\mathcal{P} ({#1})}
\newcommand{\ol}[1]{\overline{#1}}

\newcommand{\divides}{\mid}
\newcommand{\notdivides}{\nmid}

\newcommand{\st}{\text{ such that }}
\newcommand{\pmi}{\text{ principle of mathematical induction }}

\newcommand{\nitem}[1] % sets enumerate to a specific number {arg1}
{
	\setcounter{enumi}{#1}
	\addtocounter{enumi}{-1}
	\item
}
\usepackage{multicol}


\title{MATH 405: Assignment 6}
\author{Micah Sherry}
\begin{document}
	\maketitle
	\begin{enumerate}
		\item if G is a commutative Group Then $H =\{\alpha \in G |\alpha = g^2 \text{ for some } g \in G\}$ is a Subgroup of G.
		\begin{proof}
			let $G$ be a commutative Group and let $H =\{\alpha \in G |\alpha = g^2 \text{ for some } g \in G\}$.
			\\
			Notice $e = e^2 \in H$. So, $H \neq \nullset$.\\ let $x^2, y^2 \in H$
			\begin{align*}
				x^2y^2  &= xxyy\\
						&= xyxy &\text{(By commutativity of G)}\\
						&= (xy)^2
			\end{align*}
			So, $(xy)^2 \in H$ thus $H$ is closed under the operation of the group.\\
			let $g^2 \in H$ and let $(g^{-1})^2 \in H$ where $g^{-1}$ is the inverse of g.  
			Consider,
			\begin{align*}
				(g^{-1})^2g^2   &= g^{-1}g^{-1}gg \\
								&= e  &\text{(By definition of inverses)} 
			\end{align*}
			Therefore $(g^{-1})^2$ is the inverse of $g^2$. So H contains inverses for each element of H. Thus G is a Subgroup. 
		\end{proof}

		\item let H and K be Subgroup of a commutative group G. Define $HK = \{hk| h \in H \text{ and } k \in G\}$
		\begin{proof}
			Notice $e \in H $ and $e \in K$ therefore $e^2 = e \in HK$. So $HK \neq \nullset$.
			let $h_1, h_2 \in H$ and $k_1, k_2 \in K$. 
			Consider,
			\begin{align*}
				(h_1k_1)(h_2k_2)&=(h_1k_1)(h_2k_2)\\
								&= (h_1h_2)(k_1k_2) &\text{(by commutativity of G)}\\			
			\end{align*}
			Notice $(h_1h_2)\in H $ and $ (k_1k_2)$ So, $(h_1h_2)(k_1k_2) \in HK$.
			Let $h \in H$ and $k \in K$, notice $h^{-1}k^{-1} \in HK$ Consider,
			\begin{align*}
				(h^{-1}k^{-1})(hk) &= h^{-1}k^{-1}hk\\
									&= h^{-1}hk^{-1}k &\text{(by commutativity)}\\
									&= e &\text{(by definition of inverses)}
			\end{align*} 
			Therefore $h^{-1}k^{-1}$ is the inverse of $hk$. Thus $HK$ is a Subgroup of $G$.
		\end{proof}

		\item Find the order of each element in $U_{20}$
		\begin{center}
			\begin{tabular}{|c|c|}
				\hline element of $U_{20}$ & order \\ \hline\hline
				1 & 1  \\ \hline
				3 & 4  \\ \hline
				7 & 4  \\ \hline
				9 & 2  \\ \hline
				11 & 2 \\ \hline
				13 & 4 \\ \hline
				17 & 4 \\ \hline
				19 & 2 \\ \hline  
			\end{tabular}
		\end{center}
		\item Let $G$ be a group and $g \in G$ be an element with finite ord prove each of the following statements: 
		\begin{enumerate}
			\item $ord(g^{-1})$ is finite. proof of this follows from proof of part b
			\item $ord(g^{-1}) = ord(g)$ 
			\begin{proof}
				let $g \in G$ with $ord(g) = n \in \N$  
				\begin{align*}
					g^n &= e \\
					(g^n)^{-1} &= e^{-1} \\
					(g^{-1})^n &= e &\text{(by properties of exponents and since e is self inverse)}\\
				\end{align*}
				Assume for the sake of contradiction that there exist $r in \N$ such that $0 < r < n$ such that $(g^{-1})^r = e$.
				So, 
				\begin{align*}
					(g^{-1})^r &= e\\
					((g^{-1})^r)^{-1} &= e^{-1}\\
					g^r &= e &\text{(by properties of exponents and since e is self inverse)}\\
				\end{align*}
				since $0<r <n$ this implies that $ord(g)=r$ which contradicts the assumption therefore $ord(g^{-1})=n$
			\end{proof}
		\end{enumerate}
		\item let $a$ and $b$ be elements of a commutative Group g. If the $ord(a)$ and $ord(b)$ are finite then ord(ab) is finite.
		\begin{proof}
			let $a,b \in G$ with $ord(a) = m$ and $ord(b) = n$
			Notice $a^m = (a^{m})^n =a^{mn}= e^n = e$ and $b^n =(a^{m})^n = b^{nm} = e^m = e$.
			Now Consider,
			\begin{align*}
				e 	&= a^{nm}b^{nm}\\
					&= (ab)^{nm} &\text{(since G is commutative)}
			\end{align*}
			Therefore the $0 < ord(ab) <mn$, which is finite. 
		\end{proof} 
	\end{enumerate}
\end{document}